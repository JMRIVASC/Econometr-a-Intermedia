\documentclass[16.5pt]{beamer}
%\beamertemplatenavigationsymbolsempty
\usepackage[utf8]{inputenc}
\usepackage{amssymb,amsmath}
\usepackage[spanish]{babel}
\usetheme{Singapore}
%\usetheme{Copenhagen}
%\usetheme[secheader]{Boadilla}
%\usetheme{AnnArbor}
\usepackage{xcolor}
\usepackage{listings}
\usepackage{xcolor}
\usepackage{multirow,array}
\usepackage{tikz}
\usepackage{ragged2e}
\usepackage{mathtools}
\usepackage{listings}
\usepackage{color}
\usepackage{listings}
\usepackage{upquote}
\usepackage[tikz]{bclogo}

\presetkeys{bclogo}{
ombre=true,
epBord=3,
couleur = blue!15!white,
couleurBord = red,
arrondi = 0.2,
logo=\bctrombone
}{}

\definecolor{orange}{rgb}{0.03, 0.15, 0.4}
\colorlet{beamer@blendedblue}{orange!70!white}
\setbeamertemplate{navigation symbols}{}
\setbeamertemplate{caption}[numbered]
\definecolor{dkgreen}{rgb}{0,0.6,0}
\definecolor{gray}{rgb}{0.5,0.5,0.5}
\definecolor{mauve}{rgb}{0.58,0,0.82}

\lstset{frame=tb,
  language=Java,
  aboveskip=3mm,
  belowskip=3mm,
  showstringspaces=false,
  columns=flexible,
  basicstyle={\small\ttfamily},
  numbers=none,
  numberstyle=\tiny\color{gray},
  keywordstyle=\color{blue},
  commentstyle=\color{dkgreen},
  stringstyle=\color{mauve},
  breaklines=true,
  breakatwhitespace=true,
  tabsize=3
}


\title{Matrices}
\author{Juan Manuel Rivas Castillo}
\institute{UNMSM}

\date{\today}

\justifying




\titlegraphic{\vspace{0.1cm}}


\begin{document}
\maketitle 

%%%%%%%%%%%%%%%%%%%%%%%%%%%%%%%%% Matrices-Descripción %%%%%%%%%%%%%%%%%%%%%%%%%%%%%%%%%

{
\setbeamercolor{background canvas}{bg=yellow!10}
\begin{frame}[fragile, plain]
\frametitle{\textsc{\textbf{Matrices}}}
\setbeamercolor{item}{fg=blue}
\setbeamercolor{normal text}{fg=gray!40!black}
\usebeamercolor[fg]{normal text}
\hspace*{-5mm}
\vspace*{-5mm} 
\vspace{0.3cm}
\begin{itemize}
\item \texttt{ Una {\color{blue} matrix} es un arreglo rectangular de números denotados por:}
$$\color{blue} A= \mathtt{\begin{pmatrix} a_{11} &  a_{12}  &....&  a_{1k}  \\ 
a_{21} &  a_{22}  &....&  a_{2k} \\ 
... & ...&....& ... \\
... & ...&....& .... \\
a_{n1} &  a_{n2}  &....& a_{nk} \end{pmatrix}}$$

\end{itemize}
\texttt{  {\color{red} Ingrese la siguiente matriz en Rstudio} }
$$A= \mathtt{\begin{pmatrix} 1 & 5 & 6 \\ 2 & 6 & 8 \\ 7 & 1 & 8 \end{pmatrix}}$$

\begin{lstlisting}
A = matrix(c(1,5,6,2,6,8,7,1,8), nrow=3,byrow=TRUE)
\end{lstlisting}
\end{frame}


}

%%%%%%%%%%%%%%%%%%%%%%%%%%%%%%%%% Matrices-Descripción %%%%%%%%%%%%%%%%%%%%%%%%%%%%%%%%%

{
\setbeamercolor{background canvas}{bg=yellow!10}
\begin{frame}[fragile, plain]
\frametitle{\textsc{\textbf{Matrices}}}
\setbeamercolor{item}{fg=blue}
\setbeamercolor{normal text}{fg=gray!40!black}
\usebeamercolor[fg]{normal text}
\hspace*{-5mm}
\vspace*{-5mm} 
\vspace{0.3cm}

\begin{itemize}
\item \texttt{ Un {\color{blue} vector fila} es una matriz con una sola fila; mientras que un {\color{blue} vector columna} es una matriz con una sola columna  }
\item \texttt{ La {\color{blue} dimensión} de una matriz es el número de filas y columnas que esta contiene, si el número de filas es igual all número de columnas se dice que la matriz es cuadrada. Dentro de esta familia de matrices se identifican a las siguientes:}
\begin{itemize}
\item \texttt{ La {\color{blue} simétrica} aquella que es igual a su transpuesta}
\item \texttt{ La {\color{blue} diagonal} aquella cuyos elementos distintos de cero aparecen en la diagonal}
\item \texttt{ La {\color{blue} escalar} los elementos de la diagonal son iguales}
\item \texttt{ La {\color{blue} identidad} los elementos de la diagonal son unos}
\item \texttt{ La {\color{blue} triángular} si los ceros se encuentran por encima de la diagonal es triángular inferior de lo contrario triángular superior}
\end{itemize}
\end{itemize}
\end{frame}

}
%%%%%%%%%%%%%%%%%%%%%%%%%%%%%%%%% Manipulación de Matrices %%%%%%%%%%%%%%%%%%%%%%%%%%%%%%%%%

{
\setbeamercolor{background canvas}{bg=yellow!10}
\begin{frame}[fragile, plain]
\frametitle{\textsc{\textbf{Manipulación de Matrices}}}
\setbeamercolor{item}{fg=blue}
\setbeamercolor{normal text}{fg=gray!40!black}
\usebeamercolor[fg]{normal text}
\hspace*{-5mm}
\vspace*{-5mm} 
\vspace{0.3cm}

\begin{itemize}
\item \texttt{ {\color{blue} Igualdad de Matrices} Dos matrices son iguales si tienen la misma dimensión y los mismos elementos en las posiciones correspondientes  }
$\color{blue} A= B$ si y solo si $a_{ik}= b_{ik}$ para todo i y k

\end{itemize}

\begin{lstlisting}
A==A; A!=A
A>A; A<A; A>=A; A<=A
A==A & A!=A
A==A | A!=A
\end{lstlisting}
\begin{itemize}
\item \texttt{ {\color{blue} Transposición} Significa intercambiar filas por columnas}. Una matriz es simétrica si es igual a su transpuesta. $(A')'=A$. $(AB)' = B'A'$. La transpuesta de un vector fila es un vector columna.
\end{itemize}
\begin{lstlisting}
B = t(A)
B
\end{lstlisting}

\end{frame}
}

%%%%%%%%%%%%%%%%%%%%%%%%%%%%%%%%% Manipulación de Matrices %%%%%%%%%%%%%%%%%%%%%%%%%%%%%%%%%

{
\setbeamercolor{background canvas}{bg=yellow!10}
\begin{frame}[fragile, plain]
\frametitle{\textsc{\textbf{Manipulación de Matrices}}}
\setbeamercolor{item}{fg=blue}
\setbeamercolor{normal text}{fg=gray!40!black}
\usebeamercolor[fg]{normal text}
\hspace*{-5mm}
\vspace*{-5mm} 
\vspace{0.3cm}

\begin{itemize}
\item \texttt{ {\color{blue} Suma de Matrices} Para sumar matrices estas deben tener las mismas dimensiones. La matriz cero juega el mismo rol que la suma escalar}

$$\color{black} \mathtt{C = A + B=[a_{ik} + b_{ik}]}$$
$$\color{black} \mathtt{C = A - B =[a_{ik} - b_{ik}]}$$
$$\color{black} \mathtt{A + B= B + A}$$
$$\color{black} \mathtt{(A + B) + C= A +(B + C)}$$
$$\color{black} \mathtt{(A + B)'= A' +B'}$$


\end{itemize}
\end{frame}
}

%%%%%%%%%%%%%%%%%%%%%%%%%%%%%%%%% Manipulación de Matrices %%%%%%%%%%%%%%%%%%%%%%%%%%%%%%%%%

{
\setbeamercolor{background canvas}{bg=yellow!10}
\begin{frame}[fragile, plain]
\frametitle{\textsc{\textbf{Manipulación de Matrices}}}
\setbeamercolor{item}{fg=blue}
\setbeamercolor{normal text}{fg=gray!40!black}
\usebeamercolor[fg]{normal text}
\hspace*{-5mm}
\vspace*{-5mm} 
\vspace{0.3cm}

\begin{itemize}

\item \texttt{ {\color{blue} Multiplicación de Vectores} Para multplicar vectores empleamos {\color{black} producto interno } que significa multiplicar filas por columnas} $ \color{black} \mathtt{a'b = b'a}$
\item \texttt{ {\color{blue} Multiplicación de Matrices y Multiplicación Escalar} Para una matriz A de dimensión  {\color{blue} n x K} y una matriz B de dimensión  {\color{blue} K x M}, el producto matricial  {\color{blue} C = AB} es una matriz de dimensión   {\color{blue} n x M} cuyos elementos se obtienen del producto interno de las filas de A y las columnas de B.  {\color{blue} Conformables para la multiplicación}. La multiplicación de matrices por lo general es no comutativa  {\color{blue} AB puede existir pero no necesariamente BA, y si existe puede tener diferentes resultados o dimensiones}.  {\color{red} La multiplicación escalar} consiste en multiplicar cada elemento de la matriz por un escalar dado.}
\end{itemize}

\end{frame}
}
%%%%%%%%%%%%%%%%%%%%%%%%%%%%%%%%% Manipulación de Matrices %%%%%%%%%%%%%%%%%%%%%%%%%%%%%%%%%
{
\setbeamercolor{background canvas}{bg=yellow!10}
\begin{frame}[fragile, plain]
\frametitle{\textsc{\textbf{Manipulación de Matrices}}}
\setbeamercolor{item}{fg=blue}
\setbeamercolor{normal text}{fg=gray!40!black}
\usebeamercolor[fg]{normal text}
\hspace*{-5mm}
\vspace*{-5mm} 
\vspace{0.3cm}


\texttt{  {\color{red} Hallar A + B, 2A y 2A-3B} }
$$A= \mathtt{\begin{pmatrix} 1 & -2 & 3  \\ 4 & 5 & -6   \end{pmatrix}}$$
$$B= \mathtt{\begin{pmatrix} 3 & 0 & 2  \\ -7 & 1 & 8   \end{pmatrix}}$$

\begin{lstlisting}
A = matrix(c(1,-2,3,4,5,-6),nrow=2, byrow=TRUE)
B = matrix(c(3,0,2,-7,1,8), nrow=2, byrow=TRUE)
A+B
2*A
2*A-3*B
\end{lstlisting}


\end{frame}
}
%%%%%%%%%%%%%%%%%%%%%%%%%%%%%%%%% Manipulación de Matrices %%%%%%%%%%%%%%%%%%%%%%%%%%%%%%%%%
{
\setbeamercolor{background canvas}{bg=yellow!10}
\begin{frame}[fragile, plain]
\frametitle{\textsc{\textbf{Manipulación de Matrices}}}
\setbeamercolor{item}{fg=blue}
\setbeamercolor{normal text}{fg=gray!40!black}
\usebeamercolor[fg]{normal text}
\hspace*{-5mm}
\vspace*{-2mm} 
\vspace{0.3cm}


\texttt{  {\color{red} Ingrese la siguiente matriz al Rstudio y multipliquela por la matriz A inicial} }
$$B= \mathtt{\begin{pmatrix} 2 & 8  \\ 1 & 0 \\ 8 & 1  \end{pmatrix}}$$

\begin{lstlisting}
A = matrix(c(1,5,6,2,6,8,7,1,8),nrow=3,byrow=TRUE)
B = matrix(c(2,8,1,0,8,1),nrow=3,  byrow=TRUE)
C = A%*%B
\end{lstlisting}

Algunas reglas generales que se cumplen para la multiplicación de matrices
\begin{itemize}
\item\texttt{{\color{black}Ley Asociativa:} $(AB)C = A(BC)$} 
\item \texttt{{ \color{black} Ley Distributiva:} $A(B+C) = AB + AC$} 
\item \texttt{{ \color{black} Transpuesta de un Producto:} $(AB)' = B'A'$} 
\item \texttt{{ \color{black} Transpuesta de un Producto Extendido:} $(ABC)' = C'B'A'$} 

\end{itemize}
\end{frame}
}
%%%%%%%%%%%%%%%%% SUMA DE VALORES%%%%%%%%%%%%%%%%%%%
{
\setbeamercolor{background canvas}{bg=yellow!10}
\begin{frame}[fragile, plain]
\frametitle{\textsc{\textbf{Suma de Valores}}}
\setbeamercolor{item}{fg=blue}
\setbeamercolor{normal text}{fg=gray!40!black}
\usebeamercolor[fg]{normal text}
\hspace*{-5mm}
\vspace*{-5mm} 
\vspace{0.5cm}


\texttt{  {\color{blue} Sea un vector i que es una columna de unos, entonces:} }

$$\sum_{i=1}^n x_i = x_1 + x_2 + .....+ x_n = \color{black} \mathtt{i'x}$$



\begin{bclogo}{Ejercicio 1}
 {\small\color{blue} Cargar la base de datos Data1.dat en Rstudio en la cual se encuentran 5 columnas (x1, x2, x3, x4 y x5), obtenga el estadístico de suma para x1}
\begin{lstlisting}
setwd('F:\\UNMSM\\EI')
Da1 = read.csv('Data1.dat',header=FALSE, sep = "")
Unos = matrix(rep(1,dim(Da1)[1]))
x1 = as.matrix(Da1$V1)
Sx1 = t(Unos)%*%x1

\end{lstlisting}
\end{bclogo}

\end{frame}
}

%%%%%%%%%%%%%%%%%%%%%%%%%%%%%%%%% Suma de Valores %%%%%%%%%%%%%%%%%%%%%%%%%%%%%%%%%

{
\setbeamercolor{background canvas}{bg=yellow!10}
\begin{frame}[fragile, plain]
\frametitle{\textsc{\textbf{Suma de Valores}}}
\setbeamercolor{item}{fg=blue}
\setbeamercolor{normal text}{fg=gray!40!black}
\usebeamercolor[fg]{normal text}
\hspace*{-5mm}
\vspace*{-5mm} 
\vspace{0.5cm}

{\color{blue} Si todos los elementos en x son iguales a la misma constante a, entonces $x=ai$} 

$$\sum_{i=1}^n x_i = i'(ai) = a(i'i) = na$$

\texttt{  {\color{blue} Para cualquier constante a} }

$$\sum_{i=1}^n a x_i =a\sum_{i=1}^n  x_i= ai'x$$

\end{frame}
}




%%%%%%%%%%%%%%%%% P>ROMEDIO%%%%%%%%%%%%%%%%%%%


{
\setbeamercolor{background canvas}{bg=yellow!10}
\begin{frame}[fragile, plain]
\frametitle{\textsc{\textbf{Suma de Valores}}}
\setbeamercolor{item}{fg=blue}
\setbeamercolor{normal text}{fg=gray!40!black}
\usebeamercolor[fg]{normal text}
\hspace*{-5mm}
\vspace*{-2mm} 
\vspace{0.3cm}

\texttt{{\color{blue} Si a =$\frac{1}{n}$ podemos obtener la media arimética}}
\begin{tiny}
\[ \overline{x} = \frac{1}{n}\sum_{i=1}^n  x_i =\frac{1}{n}i'x \]
\end{tiny}

\begin{bclogo}{Ejercicio 2}
 {\tiny\color{blue} Empleando la base de datos anterior obtenga el promedio para x1}
\begin{lstlisting}[basicstyle=\tiny]
1/dim(Da1)[1]*t(Unos)%*%x1
\end{lstlisting}
\end{bclogo}


\texttt{  {\color{blue} Si extendemos este resultados tenemos:}}

\begin{tiny}
\[ \sum_{i=1}^n  x_i^2 = x'x; \sum_{i=1}^n  x_iy_i = x'y \]
\end{tiny}


\begin{bclogo}{Ejercicio 3}
 {\tiny\color{blue} Empleando la base de datos anterior obtenga el promedio para x1}
\begin{lstlisting}[basicstyle=\tiny]
t(x1)%*%x1; x2 = as.matrix(Da1$V2); t(x1)%*%x2
\end{lstlisting}
\end{bclogo}

\end{frame}
}

%%%%%%%%%%%%%%%%% Matriz Mo %%%%%%%%%%%%%%%%%%%%%%%%%%%%%

{
\setbeamercolor{background canvas}{bg=yellow!10}
\begin{frame}[fragile, plain]
\frametitle{\textsc{\textbf{Una Matriz Idempotente Útil}}}
\setbeamercolor{item}{fg=blue}
\setbeamercolor{normal text}{fg=gray!40!black}
\usebeamercolor[fg]{normal text}
\hspace*{-5mm}
\vspace*{-5mm} 
\vspace{0.3cm}



  {\color{blue} La matriz centrada es aquella que se emplea para transformar una data en sus desviaciones respecto a la media}

$$[x -i\overline{x}] = [x - \frac{1}{n}ii'x] =  [I - \frac{1}{n}ii']x = M^0x$$

\textbf{Matriz de Varianzas y Covarianzas}

$$S= \begin{pmatrix} var(X_1) & cov(X_1, X_2) & .... & cov(X_1, X_k) \\
cov(X_2, X_1) & var(X_2) & .... & cov(X_2, X_k) \\
............  & .......   & .... & ............ \\
cov(X_k, X_1) & cov(X_k, X_2) & .... & cov(X_k, X_k) \\
\end{pmatrix}$$

 {\color{red}  Esta matriz cuenta con las siguientes características:}
\begin{itemize}
\item Cuadrada
\item Simétrica
\item Definida Positiva
\end{itemize}

\end{frame}
}

%%%%%%%%%%%%%%%%% Matriz Varianzas y Covarianzas %%%%%%%%%%%%%%%%%%%%%%%%%%%%%

{
\setbeamercolor{background canvas}{bg=yellow!10}
\begin{frame}[fragile, plain]
\frametitle{\textbf{Varianzas y Covarianzas}}
\setbeamercolor{item}{fg=blue}
\setbeamercolor{normal text}{fg=gray!40!black}
\usebeamercolor[fg]{normal text}
\hspace*{-5mm}
\vspace*{-2mm} 
\vspace{0.3cm}

La matriz de varianzas y covarianzas poblacional (S) la obtenemos a partir de la siguiente expresión:

\tiny
$$S = X'M^0X $$
\normalsize

Tomar en cuenta que el cálculo con un conjunto de datos nos conduce a trabajar con una matriz de varianzas y covarianzas muestral.
\tiny
$$S = \frac{1}{n-1}X'M^0X $$
\normalsize

\begin{bclogo}{Ejercicio 4}
 {\tiny\color{blue} Empleando la base de datos eurosec.xlsx obtenga la matriz de varianzas y covarianzas}
\begin{lstlisting}[basicstyle=\tiny]
eurosec = read_excel("eurosec.xlsx")
X = as.matrix(eurosec[,2:10])
F = dim(X)[1]
C= dim(X)[2]
Unos = matrix(1,F,1)
Mo = diag(F)-1/F*Unos%*%t(Unos)
S = 1/(F-1)*t(X)%*%Mo%*%X
cov(X)
var(X[,1])
\end{lstlisting}
\end{bclogo}

\end{frame}
}

%%%%%%%%%%%%%%%%% Matriz de Correlaciones %%%%%%%%%%%%%%%%%%%%%%%%%%%%%

{
\setbeamercolor{background canvas}{bg=yellow!10}
\begin{frame}[fragile, plain]
\frametitle{\textbf{Matriz de Correlaciones}}
\setbeamercolor{item}{fg=blue}
\setbeamercolor{normal text}{fg=gray!40!black}
\usebeamercolor[fg]{normal text}
\hspace*{1mm}
\vspace*{-2mm} 
\vspace{0.2cm}

{\color{blue} Un problema que tiene la matriz S es que conserva las unidades de medida de las variables empleadas; por lo que una solución es la de trabajar con la matriz de correlaciones}

$$ -1 \leq Rho = \frac{cov(x,y)}{\sigma_x \sigma_y} \leq 1$$

$$R= \begin{pmatrix} 1 & Rho(X_1, X_2) & .... & Rho(X_1, X_k) \\
Rho(X_2, X_1) & 1 & .... & Rho(X_2, X_k) \\
............  & .......   & .... & ............ \\
Rho(X_k, X_1) & Rho(X_k, X_2) & .... & 1 \\
\end{pmatrix}$$

\begin{itemize}
\item Esta matriz cuenta con las siguientes características: Cuadrada, Simétrica, Traza de la matriz R = C
\item Para obtener la matriz R a partir de la matriz S realizamos la siguiente operación:
\end{itemize}

$$ R = D^{-\frac{1}{2}}SD^{-\frac{1}{2}}$$

\end{frame}
}

%%%%%%%%%%%%%%%%% Matriz de Correlaciones %%%%%%%%%%%%%%%%%%%%%%%%%%%%%

{
\setbeamercolor{background canvas}{bg=yellow!10}
\begin{frame}[fragile, plain]
\frametitle{\textbf{Matriz de Correlaciones}}
\setbeamercolor{item}{fg=blue}
\setbeamercolor{normal text}{fg=gray!40!black}
\usebeamercolor[fg]{normal text}
\hspace*{1mm}
\vspace*{-2mm} 
\vspace{0.2cm}

\begin{bclogo}{Ejercicio 5}
 {\color{blue} Empleando la matriz de covarianzas anterior obtenga la matriz de correlaciones}
\begin{lstlisting}[basicstyle=\small]
D = diag(S)
for(i in 1: C) {
    D[i]= D[i]^(-1/2)
}
D = diag(D)
D
R = D%*%S%*%D
R
cor(X)
\end{lstlisting}
\end{bclogo}
\end{frame}
}





%%%%%%%%%%%%%%%%% Determinante de una Matriz %%%%%%%%%%%%%%%%%%%%%%%%%%%%%


{
\setbeamercolor{background canvas}{bg=yellow!10}
\begin{frame}[fragile, plain]
\frametitle{\textsc{\textbf{Determinante de una Matriz}}}
\setbeamercolor{item}{fg=blue}
\setbeamercolor{normal text}{fg=gray!40!black}
\usebeamercolor[fg]{normal text}
\hspace*{-5mm}
\vspace*{-5mm} 
\vspace{0.3cm}



\texttt{  {\color{blue} El determinante de una matriz se define para matrices cuadradas. El determinante de una matriz es no cero si y solo si es de rango completo}}


$$ A= \begin{pmatrix} a & b \\ c & d \end{pmatrix}$$

$$det(A) = ad-bc$$

\begin{itemize}
\item A es inversible si y solo si $det(A) \neq 0$
\item Si $det(A) \neq  0$, entonces:

$$ A^{-1} = \frac{1}{det(A)}\begin{pmatrix} d & -b \\ -c & a   \end{pmatrix}$$

\end{itemize}

\end{frame}
}

%%%%%%%%%%%%%%%%% Autovalores y Autovectores %%%%%%%%%%%%%%%%%%%%%%%%%%%%%

{
\setbeamercolor{background canvas}{bg=yellow!10}
\begin{frame}[fragile, plain]
\frametitle{\textsc{\textbf{Autovalores y Autovectores}}}
\setbeamercolor{item}{fg=blue}
\setbeamercolor{normal text}{fg=gray!40!black}
\usebeamercolor[fg]{normal text}
\hspace*{-5mm}
\vspace*{-1mm} 
\vspace{0.3cm}

\textbf{Autovalores:} Son los valores escalares ($\lambda$) que satisfacen la ecuación característica de una matriz (A), es decir, ($A\mathbf{v} = \lambda\mathbf{v}$). Aquí, ($\mathbf{v}$) es un vector no nulo y ($\lambda$) es el autovalor correspondiente.

\textbf{Autovectores:} Son los vectores ($\mathbf{v}$) que, cuando se multiplican por la matriz (A), resultan en un múltiplo escalar de sí mismos, es decir, ($A\mathbf{v} = \lambda\mathbf{v}$). En otras palabras, la dirección del autovector no cambia tras la transformación por la matriz (A), solo su magnitud se escala por el autovalor ($\lambda$).

\begin{bclogo}{Ejercicio 6}
 {\tiny\color{blue} Empleando la matriz de correlaciones obtenga sus autovalores y autovectores; demuestre que la suma de los mismos es igual a la traza y la multiplicación es igual a su determinante}
\begin{lstlisting}[basicstyle=\tiny]
AV = eigen(R)
Autovalores = AV$values
Autovectores = AV$vectors
sum(Autovalores)
sum(diag(Autovalores))
det(R)
prod(Autovalores)
\end{lstlisting}
\end{bclogo}
\end{frame}
}

%%%%%%%%%%%%%%%%% Inversa de una Matriz %%%%%%%%%%%%%%%%%%%%%%%%%%%%%

{
\setbeamercolor{background canvas}{bg=yellow!10}
\begin{frame}[fragile, plain]
\frametitle{\textsc{\textbf{Inversa de Matrices}}}
\setbeamercolor{item}{fg=blue}
\setbeamercolor{normal text}{fg=gray!40!black}
\usebeamercolor[fg]{normal text}
\hspace*{-5mm}
\vspace*{-5mm} 
\vspace{0.3cm}

La inversa de una matriz es una matriz que, cuando se multiplica por la matriz original, da como resultado la matriz identidad. Para una matriz (A), su inversa se denota como ($A^{-1}$) y cumple la siguiente condición:

\[ A \cdot A^{-1} = A^{-1} \cdot A = I \]

donde (I) es la matriz identidad.

Condiciones para que una matriz tenga inversa:
\begin{itemize}
\item Debe ser cuadrada: El número de filas y columnas debe ser el mismo.
\item Debe ser no singular: Su determinante debe ser diferente de cero ($\det(A) \neq 0$).
\end{itemize}




\end{frame}
}
%%%%%%%%%%%%%%%%% Inversa de una Matriz %%%%%%%%%%%%%%%%%%%%%%%%%%%%%
{
\setbeamercolor{background canvas}{bg=yellow!10}
\begin{frame}[fragile, plain]
\frametitle{\textsc{\textbf{Inversa de Matrices}}}
\setbeamercolor{item}{fg=blue}
\setbeamercolor{normal text}{fg=gray!40!black}
\usebeamercolor[fg]{normal text}
\hspace*{-5mm}
\vspace*{-5mm} 
\vspace{0.3cm}
\begin{bclogo}{Ejercicio 7}
 {\color{blue} Empleando la base de datos KT.dat y considerando las explicaciones en el archivo readme.txt desarrolle el siguiente cálculo: $(X'X)^{-1}X'Y$, en el que Y es Log Hourly Wage  y en X se encuentra una columna de unos y la varaible educación y experiencia}
\begin{lstlisting}[basicstyle=\small]
KT = read.csv('KT.dat',   header=FALSE, sep = "")
Y = as.matrix(KT[,3])
KT$unos = rep(1,dim(KT)[1])
X = as.matrix(KT[,c(2:4,6)])
solve(t(X)%*%X)%*%t(X)%*%Y


\end{lstlisting}
\end{bclogo}

\end{frame}
}




\end{document}


